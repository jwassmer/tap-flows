
\documentclass[twocolumn,english,groupedaddress, superscriptaddress,aps,pre,10pt,floatfix]{revtex4-2}
\usepackage[T1]{fontenc}
\usepackage[utf8]{inputenc}
\usepackage{amsmath,amsthm,mathtools,amssymb}
\usepackage{graphicx}
\usepackage{units}
%\usepackage{dsfont}
\usepackage{xcolor}
\usepackage{soul}
\usepackage{changes}
\usepackage{xcolor} % for color
\usepackage{natbib} %citations
\bibliographystyle{unsrtnat}
\usepackage{epigraph}
\usepackage{hyperref}

\usepackage[capitalise]{cleveref}

\usepackage{url}
\usepackage{bbold} % for  Identity Matrix
\def\UrlBreaks{\do\/\do-\do_}


\newtheorem{lemma}{Lemma}
\newtheorem{thm}{Theorem}
\newtheorem{claim}{Claim}
\newtheorem{corollary}{Corollary}
\newtheorem{definition}{Definition}[section]



\newcommand{\matr}[1]{{\boldsymbol{#1}}}
\renewcommand{\vec}[1]{{\boldsymbol{#1}}}
\newcommand{\EE}{\mathfrak{E}}
\newcommand{\VV}{\mathfrak{V}}
\newcommand{\CC}{\mathfrak{C}}
\newcommand{\OO}{\mathcal{F}}
\newcommand{\eye}{\mathds{1}}
\newcommand{\LL}{\mathcal{L}}



%new commands
\newcommand{\jonas}[1]{%
  %\phantom{#1} %hide instead
  \sethlcolor{cyan}%
  \emph{\hl{[JW: #1]}}
}
\newcommand{\carsten}[1]{%
  %\phantom{#1} %hide instead
  \sethlcolor{red}%
  \emph{\hl{[CH: #1]}}
}
\newcommand{\FF}{\mathcal{F}}



\begin{document}

\title{A social cost-based centrality to quantify road significance in self-organizing traffic networks}

\author{Jonas Wassmer}
\affiliation{Potsdam Institute for Climate Impact Research (PIK), Member of the Leibniz Association, Potsdam, Germany}
\affiliation{Institute of Physics and Astronomy, University of Potsdam, Potsdam, Germany}
\author{Nils Antary}
\affiliation{Potsdam Institute for Climate Impact Research (PIK), Member of the Leibniz Association, Potsdam, Germany}
\author{Carsten Hartmann}
\affiliation{Forschungszentrum Jülich, Institute of Energy and Climate Research
– Energy System Engineering (ICE-1), 52428 Jülich, Germany}
%\author{Dirk Witthaut}
%\affiliation{Institute of Energy and Climate Research – Energy Systems Engineering (IEK-10), Forschungszentrum Jülich, 52428 Jülich, Germany}
\author{Norbert Marwan}
\affiliation{Potsdam Institute for Climate Impact Research (PIK), Member of the Leibniz Association, Potsdam, Germany}
\affiliation{Institute of Physics and Astronomy, University of Potsdam, Potsdam, Germany}


\begin{abstract}
Urban road networks are prototypical complex systems in which large numbers of individual agents interact through shared infrastructure, giving rise to collective traffic equilibria. These equilibria emerge from decentralized route choice decisions and exhibit a strong sensitivity to local changes in network properties. Even small modifications to the capacity or free-flow travel time of a single link can trigger large-scale reconfigurations of traffic flows, as illustrated by the Braess paradox. Conventional measures of link importance often overlook this systemic sensitivity and its implications for network efficiency. In this study, we introduce a social cost-based centrality measure that quantifies the marginal impact of link-level free-flow travel time perturbations on total social cost under Wardrop equilibrium. The measure is derived analytically from the linear formulation of the traffic assignment problem, enabling efficient and interpretable computation of network-wide sensitivities. We demonstrate the approach in synthetic and real-world urban networks, revealing structurally critical links that exert disproportionate influence on travel costs. Beyond improving network efficiency, this framework supports targeted interventions that can reduce unnecessary travel, lower emissions, and contribute to the transition toward more sustainable and livable cities.
\end{abstract}

\maketitle

\section{Introduction}

Urban road networks are prototypical examples of complex systems in which large numbers of heterogeneous agents (drivers) interact through shared infrastructure, giving rise to emergent macroscopic traffic states such as congestion patterns and traffic equilibria. These states arise from decentralized route choice decisions rather than centralized control, making them characteristic cases of self-organization in complex systems \citep{helbing2001traffic, nagel1992cellular, kerner2004physics, chowdhury2000statistical, helbing2009pedestrian}.

The behavior of such systems is governed by the emergence of macroscopic order from microscopic interactions. Through cooperative behavior, simple components can give rise to pattern formation, phase transitions, and system-wide stability \citep{haken1989,kirkaldy1992,camazine2001}. In traffic networks, local route choices aggregate into collective flow patterns and systemic vulnerabilities. The importance of individual roads is therefore not determined by their physical properties alone, but by their role in shaping the global organization of flows. Identifying which local components exert the greatest influence on system-wide behavior is essential for understanding, predicting, and managing these self-organizing dynamics.

A defining feature of these networks is their sensitivity to local perturbations. Even small interventions, such as modifying the capacity or free-flow travel time of a single link, can trigger large-scale reconfigurations of traffic flows. The Braess paradox illustrates this sensitivity: adding a road can paradoxically increase total travel times by shifting route choices \citep{braess1968paradoxon, roughgarden2002bad, steinberg1983prevalence}. This demonstrates that transport network performance must be understood as a systemic property, rather than a simple sum of link-level characteristics.

Wardrop \citep{wardrop1952road} formulated two fundamental principles of traffic assignment. The \emph{user equilibrium} describes the state in which no driver can reduce their travel time by unilaterally changing routes, while the \emph{system optimum} minimizes total travel time across all users. The gap between these two states reflects the inefficiency of decentralized decision-making, often quantified as the price of anarchy \citep{roughgarden2002bad, correa2008geometric, youn2008price}. Understanding how local infrastructure changes influence these efficiency gaps remains a central challenge in network science and transport research.

Previous work has addressed traffic equilibria \citep{sheffi1985urban, patriksson1994traffic, florian1995network}, link removal and capacity management \citep{nagurney2000emission, nagurney2009efficiency, xu2018critical}, and paradoxical network effects \citep{zhang1997stability, pas1997braess, zhang2018braess}. Yet, systematic and scalable methods for quantifying how marginal changes to individual links affect overall network performance are still limited. Identifying critical and paradoxical edges through sensitivity analysis can provide valuable guidance for infrastructure management and policy \citep{passacantando2021braess, altafini2022edge}.

\subsection{Motivation and contribution}

Many cities face the dual challenge of mitigating congestion while reducing emissions and reclaiming public space from car traffic \citep{newman2015end, cervero2013transport, creutzig2015transport}. Expanding road capacity has long been a standard response, but this strategy often induces additional traffic and fails to improve long-term performance \citep{downs1962law, duranton2011fundamental}. The Braess paradox provides a theoretical basis for why more capacity does not always lead to better outcomes. Addressing this challenge requires tools that can identify the network elements with the greatest systemic impact and thereby support targeted interventions such as congestion pricing, access restrictions, or adaptive signal control \citep{yang2005roadpricing, small2007economics, depalma2011congestion}.

To this end, we introduce \emph{social cost gradient centrality} (SCGC), a novel edge importance measure that quantifies how marginal changes in free-flow travel times affect total social cost. SCGC highlights both critical links that strongly influence system efficiency and Braessian links whose improvement would paradoxically worsen it.

The main contributions of this paper are:
\begin{enumerate}
    \item a closed-form matrix formulation of static traffic assignment with linear cost functions, showing that total social cost is piecewise linear in free-flow travel times,
    \item the definition of SCGC, unifying the detection of critical and paradoxical edges within a single framework,
    \item applications to synthetic and real-world urban networks, demonstrating how self-organized traffic patterns emerge from the interplay of network topology, user behavior, and link characteristics. 
\end{enumerate}

\subsection{Structure of the paper}

The remainder of this paper is structured as follows. Section~\ref{sec:model} reviews the traffic assignment model and the underlying cost function. We also revisit the Braess paradox and present an extended formulation. Section~\ref{sec:methods-solve} derives the SCGC measure analytically. Section~\ref{sec:results} applies the method to synthetic and real-world networks. Section~\ref{sec:discussion} discusses policy implications, emphasizing the identification of structurally important and paradoxical links, and concludes with an outlook on future research directions.

%\todo[inline]{here is a suggestion for linking it to Haken's work:}
%
%In understanding complex systems, the framework of synergetics, introduced 
%by Hermann Haken, emphasizes how macro-level order and patterns emerge 
%from the interactions of numerous simple, local components \cite{haken1983,haken1987}. 
%Haken's approach posits that systems self-organise through the cooperative 
%behavior of their constituents, leading to phenomena such as pattern formation, 
%phase transitions, and systemic stability-concepts that have found broad 
%application across physics, biology, and social sciences \cite{haken1989,kirkaldy1992,camazine2001}. Applying this 
%perspective to traffic networks, we recognize that the collective decisions 
%of individual drivers, while seemingly autonomous, give rise to emergent 
%traffic flow patterns, congestion hotspots, and systemic vulnerabilities. 
%The systemic importance of roads -- beyond their physical attributes -- can, 
%thus, be viewed as a manifestation of this self-organizing behavior, where 
%local interactions produce global effects. In this context, new measures like the 
%social cost-based centrality we propose serve to identify critical edges 
%that significantly influence the system's overall dynamics, aligning with 
%the synergetic principle that holistic behavior stems from micro-level 
%interactions \cite{haken1989}.




\section{Theory: Traffic modelling}\label{sec:model}

\subsection{Cost function}

Transportation network are naturally modeled as directed graphs $G = (V,E)$ where the edges $e \in E \subseteq V \times V$ correspond to the road segments. The intersections of roads are represented by nodes $n \in V$. 
If a road between two intersections $n$ and $m$ has traffic lines with opposite direction, both edges $e=(n,m)$ and $e=(m,n)$ are in the edge set $E$.\\

%In transportation networks, roads are modeled as a set of directed edges, $e \in E$, and their intersections or junctions as nodes, $i \in V$, in a graph $G(V,E)$.\\


To analyze the traffic flow $f_e \geq 0$ on a given road segment $e$ we define a cost function $t_e(f_e)$ that represents the travel time as a function of both the traffic volume and the road's capacity. Under the assumption that drivers follow basic traffic rules, for example, matching the speed of the vehicle ahead, obeying speed limits, and maintaining a minimum speed, the cost function is often modeled using the `Bureau of Public Roads' (BPR) formulation~\cite{us1964traffic} 
\begin{align}
\label{eq:bpr_general}
t_e^{(k)}(f_e) = \beta_e \left(1 + \alpha_e f_e^k \right),
\end{align}
where $\beta_e$ is the free-flow travel time, $\alpha_e$ is a parameter that captures how sensitively travel time reacts to congestion, and $k$ determines the degree of nonlinearity in the relationship.

In this work, we focus on the linear approximation of~\cref{eq:bpr_general} (i.e., $k = 1$), which simplifies the travel time function to 
\begin{align}\label{eq:linear-travel-time}
t_e(f_e) = \alpha_e f_e + \beta_e, \quad  \in [t_e^0, t_e^{\mathrm{max}}].
\end{align}
The lower bound $t_e^0$ corresponds to the minimum travel time of vehicles, which we set to walking-speed velocity, whereas the upper bound $t_e^{\mathrm{max}}$ is set by the road’s respective speed limit.



\subsection{Traffic equilibria and social cost}
\begin{figure}[ht]
    \centering
    \includegraphics[width=0.5\textwidth]{assets/fig-social-optimum-user-eq.pdf}
    \caption{Traffic flows $f_e$ in a simple network where $10$ drivers travel from node $A$ to node $D$ under two routing regimes: \textbf{(a)} user equilibrium, where each driver minimizes their individual travel time, and \textbf{(b)} social optimum, where the total system travel time is minimized. The congestion parameter is set to $\alpha_e = 1$ for all edges. The free-flow travel times are given by $\beta_{AB}=0$, $\beta_{AC}=10$, $\beta_{BD}=10$, $\beta_{BC}=1$, and $\beta_{CD}=0$.}
    \label{fig:user-eq-optimum}
\end{figure}

%We further define the total \textit{social cost} of the system as the total flow multiplied by the travel time of each vehicle, summed over all edges in the network, 
%\begin{align}\label{eq:social-cost}
%    sc(f_e) = \sum_{e\in E} f_e t_e(f_e).
%\end{align}
%This represents the total travel time experienced by all drivers, reflecting the collective burden of congestion and delays in the system.\\

%The discrepancy between the social optimum and the user equilibrium flow is an example of the price of anarchy~\cite{koutsoupias1999worst}, which measures how the efficiency of a system degrades due to the selfish behavior of its agents.
We analyze traffic dynamics by modeling each driver as a rational agent seeking to minimize their travel time between a given origin and destination. Furthermore, we assume that all drivers possess perfect, real-time knowledge of traffic conditions, effectively implying universal GPS usage. These assumptions lead to the concept of a user equilibrium which is a state in which no driver can reduce their travel time by unilaterally changing routes. This principle, also known as Wardrop's first principle~\cite{wardrop1952road}, implies that all utilized routes between an origin-destination pair experience equal and least travel times, while unused routes would have higher travel times. The resulting traffic pattern satisfies the conditions of a Nash equilibrium in a network game among trip-makers~\cite{charnes1958extremal}.

While the user equilibrium represents an individually optimal state, it does not necessarily minimize overall congestion. In contrast, when all drivers collectively minimize the total travel time across (i.e. the social cost of \cref{eq:social-cost}) the entire network, given by the social cost of the system
\begin{align}\label{eq:social-cost}
sc(f_e) = \sum_{e\in E} f_e t_e(f_e),
\end{align}
the system reaches a system optimum---a state in which traffic is distributed to achieve the most efficient network performance. This principle, also known as Wardrop's second principle~\cite{wardrop1952road}, results in an allocation where some drivers may experience longer travel times than in a user equilibrium, but the total congestion and overall delay are minimized.

The discrepancy between the social cost at user equilibrium and at system optimum highlights the inefficiencies introduced by selfish routing. This inefficiency is encapsulated by the price of anarchy~\cite{koutsoupias1999worst}, which quantifies how much the lack of coordination among selfish drivers degrades overall system performance.  In \cref{fig:user-eq-optimum}, we illustrate this by comparing the resulting traffic flows $f_e$ in a simple network where $10$ drivers travel from node $A$ to node $D$ in both: the (a) user equilibrium and the (b) social optimum. While each driver in the user equilibrium selects their route to minimize individual travel time. In contrast, the social optimum reflects coordinated routing that minimizes total travel time across all drivers. We find that the total social cost in the user equilibrium ($sc(f_e)=200$) is higher than in the social optimum ($sc(f_e)=190$), illustrating how decentralized decision-making can lead to inefficient outcomes for the system as a whole.

\subsection{The Braess paradox}

\begin{figure}[ht]
    \centering
    \includegraphics[width=0.5\textwidth]{assets/braess_social_cost.pdf}
    \caption{Illustration of the Braess paradox in a simple four-node network. (a) Total social cost of the user equilibrium as a function of the free-flow travel time $\beta_{CB}$. (b) Network topology with 40 drivers traveling from origin $A$ (red) to destination $B$ (blue), along with the cost functions of all edges. The red cross indicates the case where edge $(C,B)$ is removed ($\beta_{CB} \rightarrow \infty$).}
    \label{fig:classicl-braess}
\end{figure}

The Braess paradox describes the counterintuitive phenomenon in which adding an additional link to a transportation network can increase overall congestion and travel times~\cite{braess1968paradoxon}. This pseudo-paradox occurs in a user equilibrium scenario because individual drivers, seeking to minimize their own travel times, may inadvertently create a suboptimal state that worsens traffic flow for everyone. Similar to~\citep{schafer2022understanding, manik2022predicting}, in which the concept has been extended to electrical grids, we define the Braess paradox more broadly: a link is considered Braessian if adding capacity to it increases the social cost in the user equilibrium of the system. 
This naturally extends the classical definition, which corresponds to the case where the link initially has zero capacity.
For now, consider the linear approximation of the cost function, as shown in Eq.~\ref{eq:linear-travel-time}. Since the parameter $\beta_e$ represents the free-flow travel time on a road, we use it to determine whether an edge is Braessian: if the total social cost decreases as the free-flow travel time of a road $\beta_e$ increases, the edge $e$ is considered Braessian.

\begin{definition}[Braessian edge]
An edge $e$ in a transportation network is called \emph{Braessian} if an increase in its capacity (or equivalently, a decrease in its free-flow travel time $\beta_e$) leads to an increase in the total social cost of the system. Formally, an edge $e$ is Braessian if
\begin{align}
    \frac{\partial sc(f_e)}{\partial \beta_e} < 0.
\end{align}
\end{definition}

In \Cref{fig:classicl-braess} we illustrate the Braess paradox for a simple network topology (panel (b)), along with the corresponding edge cost functions. In this example we assume 40 drivers going from node $A$ (marked red) to node $B$ (marked blue). Panel (a) displays the total social cost of the user equilibrium of the system (see~\cref{eq:social-cost}) as a function of the free-flow travel time $\beta_{CB}$. As expected, increasing $\beta_{CB}$ from zero to approximately one initially raises the social cost. However, as $\beta_{CB}$ continues to increase, up to around 2.5, the social cost begins to decline and eventually stabilizes at a value even lower than that observed when the edge had zero free-flow travel time. The red cross marks the social cost obtained when edge $(C,B)$ is removed entirely, corresponding a free flow travel time of infinity, $\beta_{CB} \rightarrow \infty$.\\

This behavior highlights the paradoxical nature of the phenomenon. When the edge $(C,B)$ is highly efficient (i.e., when $\beta_{CB}$ is small), it attracts additional flow, causing most drivers to choose the seemingly fastest route. As a result, this route becomes heavily used, potentially leading to congestion effects. Initially, increasing the free flow travel time on this edge increases overall travel time, as one would intuitively expect. Yet, beyond a certain threshold, a paradoxical effect emerges: as $\beta_{CB}$ grows further, drivers redistribute to alternative, less congested routes, thereby reducing the total social cost to a level lower than in the initial configuration. As the edge becomes less attractive with increasing $\beta_{CB}$, the inefficient routing behavior is mitigated and the system approaches a more efficient user equilibrium. 

\subsection{The traffic assignment problem}
Having established the concepts of user equilibrium and system optimum, we now turn to the formal mathematical formulation of traffic assignment. The traffic assignment problem (TAP) seeks to determine how trips are distributed across a network under given demand and network topology. This problem can be formulated in various ways, including link-based, path-based, and cycle-based representations~\cite{ahuja1995network}, all of which are equivalent under appropriate conditions (see Appendix~\ref{app:link-path-node} for details).\\


In this work, we employ a node-link formulation of the TAP~\cite{nguyen1974algorithm}. To formally describe the network structure, we define the node-edge incidence matrix $E_{ne}$, which encodes the connectivity between nodes and edges:
\begin{align}\label{eq:edge-incidence}
    E_{ne} = 
    \begin{cases} 
    1, & \text{if edge } e \text{ originates at node } n, \\
    -1, & \text{if edge } e \text{ terminates at node } n, \\
    0, & \text{otherwise}.
    \end{cases}
\end{align}
The TAP is then stated as the optimization problem
\begin{align}\label{eq:tap-node-link}
    \min_{f_e} \quad & \sum_{e \in E} \mathcal{F}(f_e), \nonumber \\
    \text{subject to} \quad & \sum_{e \in E} E_{ne} f^w_e = p_n^w, \quad \forall n \in V,\ \forall w \in W, \nonumber \\
    & f^w_e \geq 0, \quad \forall e \in E,\ \forall w \in W, \nonumber \\
    & f_e = \sum_{w \in W} f_e^w, \quad \forall e \in E.
\end{align}
Here, $f_e^w$ denotes the flow on edge $e$ associated with the origin-destination (OD) tuple $w$, while $f_e$ represents the total flow on edge $e$, obtained by summing over all OD tuples.
We restrict each OD tuple $w$ to be either of the form $w = (o, d_1, d_2, \dots, d_n)$, where a single origin node $o$ supplies multiple destinations, or $w = (d, o_1, o_2, \dots, o_n)$, where multiple origins supply a single destination node $d$. A mixture of multiple origins and multiple destinations within the same OD tuple is not permitted, because it would break the uniqueness of directionality.
The demand corresponding to an OD tuple $w$ is represented by the OD matrix whose components are defined as:
\begin{align}
    p^w_n &=\\
    &\begin{cases}
        y^o, & \text{if node } n \text{ is an origin in } w,\\[1mm]
        y^d, & \text{if node } n \text{ is a destination in } w,\\[1mm]
        0, & \text{otherwise}.
    \end{cases}
\end{align}
Here, $y^o$ denotes the total demand injected at an origin node, and $y^d$ the total demand withdrawn at a destination node. Additionally, the demand vector $p^w$ is constructed to satisfy flow conservation: $\sum_{n \in V} p^w_n = 0.$

The objective function $\mathcal{F}(f_e)$ for the user equilibrium of the TAP is given by the cumulative travel cost experienced by all drivers on each edge, which is represented by the integration of the cost function~\cite{beckmann1956studies}
\begin{align}
    \mathcal{F}_{\text{ue}}(f_e) &= \int_0^{f_{e}} t_{e}^{(k)}(u)du.
\end{align}
In the linear approximation of the travel time function (see Eq.~\eqref{eq:linear-travel-time}), this becomes:
\begin{align}
\mathcal{F}_{\text{ue}}(f_e) = \frac{1}{2} \alpha_e f_e^2 + \beta_e f_e.
\end{align}
The system optimum, in contrast, is obtained by minimizing the total social cost across all edges (see Eq.~\ref{eq:social-cost}). Under the same linear approximation, the corresponding objective function is:
\begin{align}
\mathcal{F}_{\text{so}}(f_e) = f_e t_e(f_e) = \alpha_e f_e^2 + \beta_e f_e.
\end{align}
%\todo[inline]{add a sentence that this optimsation problem has to be solved to answer this or that question}
Solving the optimization problem in Eq.~\eqref{eq:tap-node-link} allows us to determine the equilibrium flow configuration that corresponds to either the user equilibrium or the system optimum, depending on the form of the objective function. This formulation therefore provides the analytical foundation for evaluating how changes in network structure or edge costs affect overall traffic patterns and system efficiency.

%\todo[inline]{maybe add a sentence, that in the linear approximation both optimization problems only differ by the factor $\frac{1}{2}$ in the quadratic term }
It is worth noting that, under the linear travel time approximation, the user-equilibrium and system-optimum formulations differ only by a factor of $\frac{1}{2}$ in the quadratic term of the objective function. Consequently, both problems share the same feasible region but yield distinct flow allocations due to the different weighting of congestion effects.


\section{Methods: Solving the optimization problems}\label{sec:methods-solve}
\subsection{Algebraic solution}\label{sec:tap-solution}
In this section, we derive an analytical solution for the user equilibrium of the TAP using the linear cost function specified in Equation~\ref{eq:linear-travel-time}.
To begin, we solve the problem without considering the inequality constraints, i.e., initially neglecting the non‑negativity constraint on flows. The corresponding Lagrangian function is formulated as follows:
\begin{align}\label{eq:lagrangian-deriv}
    \mathcal{L}(f_e^w, \lambda_n) &= \sum_{e \in E} \Bigl(\tfrac12\alpha_e f_e^{2}
    + \beta_e f_e\Bigr) \nonumber \\
    &\quad + \sum_{n \in V}\sum_{w \in W}\lambda_n^{w}
    \Bigl(p_n^{w} - \sum_{e \in E}E_{ne}f_e^{w}\Bigr),
\end{align}
where $\lambda_n^{w}$ denotes the Lagrange multiplier associated with node $n$ and OD‑tuple $w$.\\

To compute the optimal flows, we take the derivative of the Lagrangian with respect to $f_e^{w}$ and set it to zero:
\begin{align}\label{eq:deriv-lagrangian-zero}
\frac{\partial\mathcal{L}}{\partial f_e^{w}}
    = \alpha_e\Bigl(\sum_{w\in W}f_e^{w}\Bigr) + \beta_e
      - \sum_{n}\lambda_n^{w}E_{ne}
    \stackrel{!}{=}0,
\end{align}
where we have used the total‑flow condition $f_e=\sum_{w\in W}f_e^{w}$ from the optimisation problem Eq.~\ref{eq:tap-node-link}.\\

We can solve this equation to obtain the flows directly. However, because we have not imposed the positive‑flow constraint $f_e^{w}\ge0$, the resulting flows may contain negative values, which are physically meaningless for vehicle flow.
To address this, we now explicitly re‑introduce the non‑negativity constraint on $f_e^{w}$ from the original optimisation problem. Incorporating this constraint into Eq.~\ref{eq:lagrangian-deriv} yields:
\begin{align}
    \sum_{w\in W}f_e^{w}
    = \frac{1}{\alpha_e}\Bigl(\sum_{n\in V}\lambda_n^{w}E_{ne}-\beta_e\Bigr)
    &\ge 0 \nonumber\\
    = \frac{1}{\alpha_e}\bigl(\lambda_n^{w}-\lambda_m^{w}\bigr)
      -\frac{\beta_e}{\alpha_e} &\ge 0 .
\end{align}
From this we immediately obtain the following necessary condition for the Lagrange multipliers $\lambda_n^{w}$:
\begin{align}\label{eq:lambda-condition}
    \lambda_n^{w}\;\ge\;\lambda_m^{w}+\beta_e ,
\end{align}
which implies that, for each OD‑tuple $w$, the multiplier at any node $n$ must be at least as large as the multiplier at its neighbouring node $m$, plus the connecting edge length $\beta_e$.\\
\begin{figure}[ht]
    \centering
    \includegraphics[width=0.5\textwidth]{assets/fig0-lambda-greater.pdf}
    \caption{Illustration of the directional-consistency condition imposed by~\cref{eq:lambda-condition}. Along each directed edge, the Lagrange multipliers must increase by at least the corresponding edge length $\beta_e$. Consequently, the graph cannot contain directed loops when all edge lengths are strictly positive.}
    \label{fig:lambda-condition}
\end{figure}

The implications of this condition are illustrated in \cref{fig:lambda-condition}. Starting from a node~$n$, the multiplier of a neighbouring node~$m$ satisfies $\lambda_m=\lambda_n+\beta_{nm}$. If a directed cycle existed, returning to the initial node would imply
\begin{align}
    \lambda_n = \lambda_n + \beta_{nm} + \beta_{mn},
\end{align}
which contradicts the original condition. Hence, if both edge costs $\beta_{nm}$ and $\beta_{mn}$ are positive, a consistent algebraic solution cannot exist. Feasible solutions are therefore restricted to graph structures that satisfy \cref{eq:lambda-condition}, i.e. to \emph{directed acyclic graphs (DAGs)}.

We can circumvent this limitation by decomposing the original graph into a set of \emph{flow subgraphs}. Consider an origin–destination tuple $w=(o,d_1,d_2,\dots,d_{|V|})$. Since there is a single origin node and all remaining nodes act as destinations, the set of edges used by travellers associated with $w$ cannot form directed cycles. Each such subgraph is therefore a DAG by construction.

\begin{figure*}[ht]
    \centering
    \includegraphics[width=\textwidth]{assets/multicom-fig.pdf}
    \caption{Decomposition of flows in an example graph under a positive‑flow constraint. \textbf{(a)} Total flow $f_e$, composed of the individual flow components $f_e^{w}$ shown in \textbf{(b)}–\textbf{(e)}. Each flow‑component subgraph $G^{w}(V,E^{w})$ also displays its respective OD‑matrix components $p_n^{w}$ colour‑coded on the nodes.}
    \label{fig:pos-flow-comp}
\end{figure*}

Repeating this construction for multiple origin–destination pairs, where each node serves as an origin exactly once, naturally leads to a decomposition of the overall flow into subgraphs $G^{w}(V,E^{w})$. Each subgraph shares the same node set $V$ but includes only edges with positive flow values (so that \cref{eq:lambda-condition} is satisfied), i.e. $E^{w}\subseteq E$. Consequently, every OD‑tuple $w$ defines its own flow subgraph, ensuring that all flow values remain non‑negative. The effective edge‑incidence matrix for each subgraph can be expressed as
\begin{align}\label{eq:subgraph-edge-incidence}
    E^{w}_{ne}=
    \begin{cases}
        1  & \text{if edge }e\text{ originates at node }n\text{ and }f_e^{w}>0,\\
       -1  & \text{if edge }e\text{ terminates at node }n\text{ and }f_e^{w}>0,\\
        0  & \text{otherwise.}
    \end{cases}
\end{align}

By considering these subgraphs we can reformulate the system using a generalized node‑edge incidence matrix, where rows corresponding to edges that do not exist in subgraph $w$ are omitted:
\begin{align}\label{eq:generalized-incidence}
    \matr{\mathcal{E}}=
    \begin{bmatrix}
        \matr{E}^{1} & \matr{0} & \cdots & \matr{0}\\
        \matr{0} & \matr{E}^{2} & \cdots & \matr{0}\\
        \vdots   & \vdots   & \ddots & \vdots\\
        \matr{0} & \matr{0} & \cdots & \matr{E}^{W}
    \end{bmatrix}
    \in\mathbb{R}^{\bigl(|E^{1}|+|E^{2}|+\dots+|E^{W}|\bigr)\times |V|\cdot|W|}.
\end{align}
The total numbers of edges and nodes in all subflows are
\begin{align}
    N_{t,e} &= \sum_{w\in W}|E^{w}|\le |E|\cdot|W|,\\
    N_{t,n} &= \sum_{w\in W}|V^{w}| = |V|\cdot|W|.
\end{align}
For convenient notation we introduce the function $e(i)$, which returns the index of the edge in the original graph that corresponds to the edge in the subgraph referred to by $i$ in the matrix $\matr{\mathcal{E}}$. 

This provides a structured way to represent the disaggregated flow components. If the subgraph structure is known, the problem can be solved immediately using Eq.~\ref{eq:deriv-lagrangian-zero}. The subgraph structure itself can be determined numerically by solving the problem once, after which the analytical formulation can be applied directly. This decomposition is particularly useful because it enables an analytical solution while preserving the non‑negativity of flows.
Thus, we can adjust the parameters $\alpha_e$ and $\beta_e$ of the system to generate similar solutions. The advantage of this approach is that it allows us to obtain multiple solutions while requiring a numerical solver only once.
The approach relies on the assumption that small variations in these parameters do not alter the subgraph structure, ensuring the validity of the analytical solution.\\

In Fig.~\ref{fig:pos-flow-comp} we illustrate the traffic flow on a small graph under a positive‑flow constraint. Panel (a) depicts the total flow $f_e$, while panels (b)–(e) show the corresponding flow decompositions for each source $w$. To determine the individual subgraph structures $G^{w}(V,E^{w})$, we first solve the full system numerically using the MOSEK or OSQP API for Python~\cite{mosek,stellato2020osqp} to obtain the respective Lagrange multipliers $\lambda_n^{w}$. In this example we consider four sources ($W=4$), leading to four subgraphs, which are shown in panels (b)–(e). Combining these four subgraphs reconstructs the total flow $f_e$ shown in panel (a).\\

%%%%%%%%%%%%%%%%%%%
\subsection{Matrix formulation of the TAP}

In order to prove in later sections that the social cost at user equilibrium is linear in the free‑flow travel time $\beta_e$, we first reformulate the optimality conditions of \cref{eq:deriv-lagrangian-zero} in matrix form. This allows us to express the TAP as a linear system of equations, which is convenient for subsequent analytical steps.

We start by incorporating the flow‑conservation constraint of the TAP,
\begin{align}
   \sum_{e\in E}E_{ne}f_e^{w}=p_n^{w},
\end{align}
and combine it with \cref{eq:deriv-lagrangian-zero} to obtain
\begin{align}
\matr{M}
\begin{pmatrix}
\vec{f}\\ \boldsymbol{\lambda}
\end{pmatrix}
=
\begin{pmatrix}
-\boldsymbol{\beta}\\ \mathbf{p}
\end{pmatrix}.
\end{align}
where the vector $\vec{f}$ collects the flow variables $f_e^{w}$ and the vector $\boldsymbol{\lambda}$ collects the Lagrange multipliers $\lambda_n^{w}$ for all nodes and OD‑tuples $w$.
The flow vector $\vec{f}$ is defined as
\begin{align}
\vec{f}=
\begin{pmatrix}
\vec{f}^{1}&\vec{f}^{2}&\dots&\vec{f}^{|W|}
\end{pmatrix}^{\!\top}
\in\mathbb{R}^{N_{t,e}}.
\end{align}
The entries of $\vec{f}$ are ordered by stacking the flows for each OD‑tuple $w$ consecutively; the first $|E^{1}|$ elements correspond to subgraph $G^{1}(V,E^{1})$, the next $|E^{2}|$ to $G^{2}(V,E^{2})$, and so on (compare with~\cref{fig:pos-flow-comp}, panels (b)–(e)).

The vector of Lagrange multipliers $\boldsymbol{\lambda}$ is defined as
\begin{align}
\boldsymbol{\lambda}=
\begin{pmatrix}
\vec{\lambda}^{1}&\vec{\lambda}^{2}&\dots&\vec{\lambda}^{|W|}
\end{pmatrix}^{\!\top}
\in\mathbb{R}^{N_{t,n}},
\end{align}
where $\vec{\lambda}^{w}$ contains the multipliers for subgraph $G^{w}(V,E^{w})$. The ordering follows the same stacking convention as for $\vec{f}$.

Analogously, we define the right‑hand‑side vectors: $\boldsymbol{\beta}\in\mathbb{R}^{N_{t,e}}$ contains the edge‑cost coefficients $\beta_i=\beta_{e(i)}$, and $\mathbf{p}\in\mathbb{R}^{N_{t,n}}$ contains the nodal demands $p_n^{w}$.

Finally, the coefficient matrix $\matr{M}$ has the block structure
\begin{align}
\matr{M}=
\begin{pmatrix}
\matr{K}&\matr{\mathcal{E}}^{\!\top}\\
\matr{\mathcal{E}}&\matr{0}
\end{pmatrix}
\in\mathbb{R}^{\bigl[N_{t,e}+N_{t,n}\bigr]\times\bigl[N_{t,e}+N_{t,n}\bigr]},
\end{align}
where $\matr{\mathcal{E}}$ is the stacked edge–node incidence matrix introduced in \cref{eq:generalized-incidence}. Furthermore, the matrix $\mathbf{K}\in\mathbb{R}^{N_{t,e}\times N_{t,e}}$ is defined as
\begin{align}
\mathbf{K}=
\begin{cases}
\alpha_{e(j)} & \text{if }e(l)=e(j),\\[2pt]
0             & \text{otherwise.}
\end{cases}
\end{align}
and $\matr{0}$ denotes the zero matrix.\\

For many configurations this problem is underdetermined, because only the total flows on the edges matter, which allows for multiple configurations of how the subflows create these optimal total flows. Furthermore, the solutions depend only on differences in $\lambda$ values, so one reference $\lambda$ must be fixed for each subgraph. Finally, the condition that every subgraph must have the same inflow and outflow is also fixed by the structure of the equations. In order to obtain an invertible problem, each of these issues must be addressed.

To fix how the subflows combine into the total flows, the most straightforward way is to minimise the square of the individual flows; this favours an even share between different flows on the individual edges. This is done by adding a small value $\delta$ to the diagonal of $\matr{K}$. This $\delta$ should be much smaller than all $\alpha$, so it only resolves the problem of multiple solutions without changing the total flows.

The $\lambda$ value of the single source for each origin‑destination tuple is set to zero, and the continuity equation for the same node in this subgraph is removed because it is redundant.

This gives the resulting matrix,
\begin{align}
\tilde{\matr{M}} &=
\begin{pmatrix}
\tilde{\matr{K}} & \tilde{\matr{\mathcal{E}}}^{\!\top}\\
\tilde{\matr{\mathcal{E}}} & \matr{0}
\end{pmatrix}
\in\mathbb{R}^{\bigl[N_{t,e}+N_{t,n}-1\bigr]\times\bigl[N_{t,e}+N_{t,n}-1\bigr]},\\
\tilde{\matr{K}} &= \matr{K}+\delta\mathbb{1}.
\end{align}
This matrix is invertible and the solutions for the flows and $\lambda$’s can therefore be written as
\begin{align}\label{eq:flow-solution-mat}
\begin{pmatrix}
\vec{f}\\ \tilde{\boldsymbol{\lambda}}
\end{pmatrix}
= \tilde{\matr{M}}^{-1}
\begin{pmatrix}
-\boldsymbol{\beta}\\ \tilde{\mathbf{p}}
\end{pmatrix}.
\end{align}

To simplify the notation, the tildes are omitted from here on. 
Using the Schur complement,
\begin{align}\label{eq:schur-complement}
\matr{S}= -\matr{\mathcal{E}}\matr{K}^{-1}\matr{\mathcal{E}}^{\!\top},
\end{align}
which corresponds to the negative of the Laplacian for the system of subgraphs. Then we can write
\begin{align}\label{eq:M-mat}
\matr{M}^{-1}=
\begin{pmatrix}
\matr{K}^{-1}+\matr{K}^{-1}\matr{\mathcal{E}}^{\!\top}\matr{S}^{-1}\matr{\mathcal{E}}\matr{K}^{-1}
    & -\matr{K}^{-1}\matr{\mathcal{E}}^{\!\top}\matr{S}^{-1}\\[4pt]
-\matr{S}^{-1}\matr{\mathcal{E}}\matr{K}^{-1}
    & \matr{S}^{-1}
\end{pmatrix}.
\end{align}
Using this matrix formulation we provide a compact algebraic representation of the TAP solution, which will be particularly useful for deriving analytical properties of the equilibrium solution in subsequent sections.
%%%%%%%%%%%%%%%%%%%%%%5

\subsection{Linearity of social cost}
\begin{lemma}[Linearity of Social Cost in User Equilibrium]\label{lemma:linear}
In the traffic assignment problem, the social cost at user equilibrium is a (piecewise) linear function of the free‑flow travel times $\beta_e$.
\end{lemma}

\begin{proof}
First we note that the total social cost is given by
\begin{align}
    sc &= \sum_{e} f_e t_e\\
       &= \sum_{e} f_e\bigl(\alpha_e f_e+\beta_e\bigr)\\
       &= \sum_{w}\sum_{e} f_e^{w}\bigl(\alpha_e f_e+\beta_e\bigr).
\end{align}
From \cref{eq:deriv-lagrangian-zero} we know that this is equal to
\begin{equation}
    sc = \sum_{w}\sum_{e} f_e^{w}\bigl(\lambda_{o(e)}-\lambda_{d(e)}\bigr).
\end{equation}
where $\lambda_{o(e)}$ and $\lambda_{d(e)}$ mark the origin and destination nodes of edge $e$.
By noting that the vector $p$ contains the flow starting at the unique origin node of subgraph $w$, $o(w)$, and ending at another node $n$, and that the total travel time is independent of the path taken, we have
\begin{equation}
\begin{split}
t^{w}_{\text{total},n}
    &= \sum_{e\in\text{path}(o(w),n)} t_e\\
    &= \sum_{e\in\text{path}(o(w),n)}\bigl(\lambda^{w}_{o(e)}-\lambda^{w}_{d(e)}\bigr)\\
    &= \lambda^{w}_{o(w)}-\lambda^{w}_{n}.
\end{split}
\end{equation}
Therefore
\begin{equation}
    sc = \sum_{w}\sum_{n} p^{w}_{n}\,t^{w}_{\text{total},n}
        = \sum_{w}\sum_{n}p^{w}_{n}\bigl(\lambda^{w}_{o(w)}-\lambda^{w}_{n}\bigr).
\end{equation}
Here $p$ is the reduced vector that contains only the destination values. Using the stacked notation from before and the fact that we set $\lambda^{w}_{o(w)}=0$, we obtain
\begin{equation}
    sc = -\sum_{i} p_{i}\lambda_{i}.
\end{equation}
With \cref{eq:flow-solution-mat} this can be rewritten as
\begin{equation}
    sc = -\sum_{i} p_{i}\bigl(\matr{S}^{-1}\matr{\mathcal{E}}\matr{K}^{-1}\boldsymbol{\beta}
                     + \matr{S}^{-1}\mathbf{p}\bigr)_{i}.
\end{equation}
Hence $sc$ is linear in every $\beta_{k}$. The derivative of $sc$ with respect to $\beta_{k}$ is
\begin{equation}\label{eq:cost-gradient}
    \frac{\partial sc}{\partial \beta_{k}}
        = -\,p_{q}\,S^{-1}_{qp}\,\mathcal{E}_{pl}\,K^{-1}_{lj}\,\delta_{e(j),k},
\end{equation}
where summation over repeated indices is implied.
The analytical expression for $K^{-1}_{lj}$ is
\begin{align}
K^{-1}_{lj}= \frac{1}{\delta}
\begin{cases}
1-\dfrac{\alpha_{e(j)}}{\delta+\alpha_{e(j)}\,n(e(j))}, & l=j,\\[6pt]
-\dfrac{\alpha_{e(j)}}{\delta+\alpha_{e(j)}\,n(e(j))}, & e(l)=e(j)\ \text{and}\ l\neq j,\\[6pt]
0, & \text{otherwise.}
\end{cases}
\end{align}
where $n(e(i))$ is the number of subflows that include edge $e(i)$. Defining $L_{lj}= \delta K^{-1}_{lj}$, \cref{eq:cost-gradient} becomes
\begin{equation}
    \frac{\partial sc}{\partial \beta_{k}}
        = p_{q}\bigl(\mathcal{E}L\mathcal{E}^{\!\top}\bigr)^{-1}_{qp}\,
          \mathcal{E}_{pl}\,L_{lj}\,\delta_{e(j),k}.
\end{equation}
Writing it this way removes the prefactor $1/\delta$, which makes the step‑wise calculation on a computer more accurate, even for very small $\delta$.
\end{proof}


% OLD VERSION
% To show that the social cost depends linearly on $\beta_e$, we take its derivative with respect to $\beta_e$.

% Substituting the linear cost function~\cref{eq:linear-travel-time} into the definition of the social cost~\cref{eq:social-cost} yields the following expression for the social cost:
% \begin{equation}
%     sc = \sum_{e} \left( \alpha_e f_e^2 + \beta_e f_e \right).
% \end{equation}
% Using \cref{eq:deriv-lagrangian-zero}, the equilibrium flow can be expressed as:
% \begin{equation}
%     f_e = \sum_w f_e^w = \frac{1}{\alpha_e} 
%     \left( \sum_n \lambda^w_n E^w_{ne} - \beta_e \right),
% \end{equation}
% where $e$ is the index of an edge in subgraph $w$.\\

% Substituting this into the expression for social cost (see~\cref{eq:social-cost}) yields the social cost at user equilibrium:
% \begin{align}
%     \begin{split}
%         sc &= \sum_{e} \frac{1}{\alpha_e} \Biggl( \left(
%         \sum_n \lambda^w_n E^w_{ne} - \beta_e \right)^2 \\
%         &+ \beta_e \left( \sum_n \lambda^w_n E^w_{ne} - \beta_e \right) \Biggl) \\
%         &= \sum_{e} \frac{1}{\alpha_e} \left(
%         \left( \sum_n \lambda^w_n E^w_{ne} \right)^2 
%         -  \beta_e \left( \sum_n \lambda^w_n E^w_{ne} \right)
%         \right).
%     \end{split}
% \end{align}
% Here, the index $w = w(e)$ depends on $e$ because for every edge $e$, we must pick a corresponding subgraph $G(V, E^w)$ that contains the edge $e$. 

% Now, we change the index notation and replace $w$ and $e$ with a unified index $i$, following the same approach used in the matrix notation. All relevant quantities are stacked into vectors. Using this notation, we can rewrite the social cost as:
% \begin{equation}
%     sc = \sum_{i} \frac{1}{n_i \alpha_i} \left(
%         \left( \sum_j \lambda_j \mathcal{E}_{ji} \right)^2 
%         -  \beta_i \left( \sum_j \lambda_j \mathcal{E}_{ji} \right)
%         \right)
% \end{equation}
% where $\alpha_i$, $\beta_i$, and $\lambda_i$ are the stacked versions of $\alpha_e$, $\beta_e$, and $\lambda_n$, respectively. Similarly, $\mathcal{E}_{ji}$ represents the stacked incidence matrices, and for a system with $W$ subgraphs, this stacked incidence matrix takes the same block diagonal form as in \cref{eq:generalized-incidence}.
% To avoid double counting edges that appear in multiple subgraphs, we normalize by the number of subgraphs $n_i$ in which each edge $i$ appears.\\


% Then, the derivative with respect to $\beta_k$ can be calculated as:
% \begin{align}\label{eq:derivative_sc}
%     \begin{split}
%         \frac{\partial sc}{\partial \beta_k} &= 
%         \sum_{i} \frac{1}{n_i \alpha_i} \left(
%         2 \left( \sum_j \lambda_j \mathcal{E}_{ji} \right) 
%         \left( \sum_l \frac{\partial \lambda_l}{\partial \beta_k} \mathcal{E}_{li} \right) \right. \\
%         &+ \left.  \beta_i \left( \sum_j \frac{\partial \lambda_j}{\partial \beta_k} \mathcal{E}_{ji}  \right) \right)
%         + \frac{1}{\alpha_k n_k} \left( \sum_j \lambda_j \mathcal{E}_{jk} \right)
%         \end{split}
% \end{align}
% From \cref{eq:flow-solution-mat} we can compute the stacked Lagrange multipliers as:
% \begin{equation}
%     \lambda_j = \sum_i -C_{ji} \beta_i + \sum_l D_{jl} p_l
%     \label{eq:lambda}
% \end{equation}
% Where $\matr{C}$ and $\matr{D}$ are the two bottom blocks of the inverse matrix $\matr{M}^{+}$ from \cref{eq:M-mat}, 
% \begin{align}
%     \begin{split}
%     \matr{C} &= (\matr{\mathcal{E}} \matr{K}^{+} \matr{\mathcal{E}}^\top)^{+} \matr{\mathcal{E}} \matr{K}^{+} \\
%     \matr{D} &= (\matr{\mathcal{E}} \matr{K}^{+} \matr{\mathcal{E}}^\top)^{+}.
%     \end{split}
% \end{align}
% Then, the derivative of the Lagrange multipliers with respect to $\beta_k$ is given by:
% \begin{equation}
%     \frac{\partial \lambda_i}{\partial \beta_k} = -C_{ik}
%     \label{eq:derivative_lambda}
% \end{equation}
% Furthermore, the factor $\frac{1}{n_i \alpha_i}$ can be rewritten as the sum over a matrix with:
% \begin{equation}
%     \frac{1}{n_i \alpha_i} = \sum_j \begin{cases}
%         \frac{1}{\alpha_i n_i} & \text{if } i = j \\
%         0 & \text{else}
%     \end{cases}
% \end{equation}\todo{sollten die indizes in der summe dann nicht $j$ sein?}
% This diagonal matrix can be identified as a pseudo--inverse of the matrix $\matr{K}$,
% \begin{equation}
%     K^{+}_{ij} = \begin{cases}
%         \frac{1}{\alpha_i n_i} & \text{if } i = j \\
%         0 & \text{else}
%     \end{cases}
%     \label{eq:K_inv}
% \end{equation}
% Substitute \cref{eq:lambda,eq:derivative_lambda,eq:K_inv} into \cref{eq:derivative_sc} and use Einstein notation to get:
% \begin{equation}
%     \begin{split}
%     \frac{\partial sc}{\partial \beta_k} &= 2 K^{+}_{iq} (-C_{js} \beta_s + D_{jl} p_l) \mathcal{E}_{ji} (-C_{rk} \mathcal{E}_{rq}) \\
%         &+  K^{+}_{iq} \beta_q (- C_{jk} \mathcal{E}_{ji}) + K^{+}_{kq} (-C_{js} \beta_s + D_{jl} p_l) \mathcal{E}_{jq}
%     \end{split}
% \end{equation}
% Using the fact that $(K^{+})^\top = K^{+}$ this can be reordered and rewritten to:
% \begin{equation}
%     \begin{split}
%     \frac{\partial sc}{\partial \beta_k} &= 2  C_{kr}^\top \mathcal{E}_{rq} K^{+}_{qi} \mathcal{E}_{ij}^\top (C_{js} \beta_s - D_{jl} p_l)\\
%         &- C_{kj}^\top \mathcal{E}_{ji}  K^{+}_{iq} \beta_q - K^{+}_{kq} \mathcal{E}_{qj}^\top (C_{js} \beta_s - D_{jl} p_l) 
%     \end{split}
% \end{equation}
% Switching to matrix--vector notation this yields:
% \begin{equation}
%     \begin{split}
%     \frac{\partial sc}{\partial \beta_k} &= \left(2  \matr{C}^\top \matr{\mathcal{E}} \matr{K}^{+} \matr{\mathcal{E}}^\top (\matr{C} \vec{\beta} - \matr{D} \vec{p}) \right.\\
%         & \left. - \matr{C}^\top \matr{\mathcal{E}}  \matr{K}^{+} \vec{\beta} - \matr{K}^{+} \matr{\mathcal{E}}^\top (\matr{C} \vec{\beta} - \matr{D} \vec{p}) \right)_k
%     \end{split}
%     \label{eq:derivative_sc_mat}
% \end{equation}
% Next we compute the transposed of the matrices $\matr{C}$ and $\matr{D}$:
% \begin{equation}
%     \begin{split}
%         \matr{C}^\top &= \matr{K}^{+} \matr{\mathcal{E}}^\top (\matr{\mathcal{E}} \matr{K}^{+} \matr{\mathcal{E}}^\top)^{+} \\
%         \matr{D}^\top &= (\matr{\mathcal{E}} \matr{K}^{+} \matr{\mathcal{E}}^\top)^{+}
%     \end{split}
%     \label{eq:schur}
% \end{equation}
% \todo[inline]{erkläre warum transpose so richtig sind. Symmetrie von $K$ -> $K^{+}$ -> $(E K^{+} E^\top)^{+}$ }
% Inserting \cref{eq:schur} into \cref{eq:derivative_sc_mat} and simplifying gives:
% \begin{equation}\label{eq:cost-gradient}
%     \begin{split}
%         \frac{\partial sc}{\partial \beta_k} &= - (\matr{C}^\top \vec{p})_k \\
%         &= -\frac{1}{n_k \alpha_k}\sum_{j,l} \mathcal{E}_{jk} D_{jl} p_l = \frac{1}{n_k \alpha_k} \sum_{l} (D_{k[0] l} - D_{k[1] l} )p_l
%     \end{split}
% \end{equation}
% which is independent of $\beta_e$ and proving that the social cost is linear in free-flow travel time.
% \end{proof}


In this proof, we utilized the algebraic solution of the user equilibrium flows, given by \cref{eq:flow-solution-mat} for the optimization problem in \cref{eq:tap-node-link}, to compute the derivative of the social cost with respect to the edges free-flow travel time $\beta_e$. We demonstrated that this derivative remains constant in terms of $\beta_e$. This property allows us to determine whether the social cost increases or decreases with respect to $\beta_e$, thereby assessing whether individual edges contribute to the system's efficiency or inefficiency.\\

However, there are certain caveats: we must first solve the system numerically once to identify subgraphs $G(V,E^w)$ where an algebraic solutions exist. The linearity of the social cost can only be established within this subgraph. If $\beta_e$ changes too significantly, the underlying subgraph solution may shift, altering the validity of our algebraic solution. Consequently, this approach can be interpreted as analyzing the system's behavior under infinitesimal changes in $\beta_e$, i.e. we can only prove piecewise linearity.


\subsection{Social cost gradient centrality}\label{sec:scgc}

We refer to the newly introduced measure defined in \cref{eq:cost-gradient} as \textit{social cost gradient centrality} (SCGC). It quantifies the sensitivity of the total social cost to incremental variations in the free-flow travel time parameter $\beta_e$. By explicitly capturing how marginal adjustments to edge characteristics influence network-wide performance, SCGC ranks edges according to their functional significance in maintaining, improving, or impairing the system’s efficiency.\\

Edges with high SCGC values are critical for overall network performance, as increases in their free-flow travel time would lead to substantial rises in total social cost. Such edges form the structural backbone of the network and require careful operational management. In contrast, edges with low SCGC values have little influence on aggregate efficiency, making them flexible candidates for reallocation, traffic calming, or other interventions. Finally, edges with negative SCGC values correspond to \textit{Braess edges}: decreasing their free-flow travel time would paradoxically increase total system cost. This offers a direct and quantitative criterion for detecting Braess-like behavior in large-scale networks without relying on stylized examples or exhaustive scenario testing.\\

Classical formulations of the Braess paradox \cite{braess1968paradoxon} identify paradoxical behavior through specific network topologies or through numerical experiments that remove edges and recompute equilibria. SCGC generalizes this concept: rather than requiring $|E|+1$ equilibrium computations, we obtain the gradient information analytically after solving the traffic assignment problem once. This makes the method computationally tractable even for large real-world networks.\\

Leveraging \cref{lemma:linear}, the SCGC can be computed efficiently, as it only requires the pseudo-inverse of the Schur complement of the KKT system associated with the linearized problem. The measure provides immediate insight into whether small increases in edge cost parameters improve or worsen system efficiency. It is important to note that the method assumes the effective subgraph $G^w(V,E^w)$ remains unchanged under small variations in $\beta_e$. While valid for marginal perturbations, this assumption may not hold for larger interventions such as complete edge removal. Therefore, SCGC provides a reliable local indicator for identifying Braess edges and other critical links, while larger structural changes may require additional equilibrium computations.


%%%%%%%%%%%%%%%%%%%%%%%%%%
%%%%%%%%%%%%%%%%%%%%%%%%%%





%\newpage
\section{Results: Applications to real world road networks}\label{sec:results}
In the following, we apply the SCGC framework to both synthetic and real-world urban road networks to demonstrate its computational efficiency and interpretive value.
\subsection{Validating social cost gradient centrality}
\begin{figure*}[ht] 
\centering \includegraphics[width=\textwidth]{assets/classic-braess-social-cost.pdf} \caption{Derivative of the social cost with respect to $\beta_e$ for each edge in a small example graph. \textbf{(a)} Analytical solution compared to \textbf{(b-f)} numerical approximations. Positive values (red) indicate that increasing $\beta_e$ raises the social cost, while negative values (blue) suggest a decrease, marking the edge as Braessian.} \label{fig:braess-social-cost} 
\end{figure*}

To validate our approach, \cref{fig:braess-social-cost} illustrates the derivative of social cost with respect to $\beta_e$ for each edge in a small example graph. Panel (a) shows the analytical solution (see \cref{eq:cost-gradient}). Panels (b–f) display numerical evaluations obtained through multiple numerical solutions—specifically, solving the system 25 times per edge, amounting to 100 solutions in total for this small example. The analytical method thus significantly reduces computational effort.

Edges highlighted in red exhibit positive derivatives, indicating that increasing their costs leads to higher social costs. In contrast, edges highlighted in blue show negative derivatives, implying that raising their costs reduces total social cost, thereby classifying them as Braessian.

Panels (b–f) further illustrate that the analytical and numerical solutions coincide within a neighborhood of the initial $\beta_e$ values (highlighted in the corresponding colors), where the relationship of the social cost and $\beta_e$ is piecewise linear. As $\beta_e$ deviates further from its baseline, discrepancies arise due to structural changes in the effective subgraph, underscoring that the analytical approximation is reliable primarily for small perturbations of $\beta_e$.

Additionally, in panels (b–e) we demonstrate that the SCGC is positive, meaning that increasing the corresponding $\beta_e$ values results in higher overall social cost. In contrast, panel (f) presents an edge with a negative SCGC, indicating that an increase in $\beta_e$ for this edge would reduce social cost, revealing a Braessian effect.











\subsection{Ranking edges by social cost gradient centrality}
\begin{figure*}[ht]
\centering
\includegraphics[width=\textwidth]{assets/synthetic-gamma-scan.pdf}
\caption{Visualization of \textbf{(a)} traffic flows and \textbf{(b)} SCGC on a synthetic planar graph $G(V,E)$ with $|V| = 50$ nodes and $|E| = 270$ edges. \textbf{(c)} Scatter plot comparing traffic flows and SCGC values.}
\label{fig:cost-gradient}
\end{figure*}
The SCGC quantifies the sensitivity of the total social cost with respect to changes in an edge’s free-flow travel time $\beta_e$ (e.g., due to speed limit adjustments). High SCGC values indicate edges where small increases in travel time result in disproportionately large increases in total social cost, highlighting their systemic importance.

To illustrate this concept, \cref{fig:cost-gradient} presents traffic flows and SCGC values for a synthetic planar graph $G(V,E)$ with $|V| = 50$ nodes and $|E| = 270$ edges, representing a stylized urban road network. The cost function parameter $\alpha_e$ (see \cref{eq:linear-travel-time}) was sampled uniformly from the interval $[0.1, 1]$. The free-flow travel time parameter $\beta_e$ was determined from the geodesic length of each edge, assuming a uniform speed limit of $50,\mathrm{km/h}$.

Next, we assigned node populations according to a power-law distribution with an exponent of $-1$, resulting in a few highly populated nodes and many nodes with relatively small populations—reflecting typical real-world urban network structures. Based on these populations, we constructed an origin–destination (OD) matrix in which a fraction of each node’s population travels to other nodes.

In this formulation, we assume that the likelihood of drivers originating from or traveling to a node is determined solely by its population. As a result, the probability of traveling between any two nodes is independent of the distance separating them. While this assumption simplifies the model, a more realistic representation could incorporate distance effects, for example by using a gravity model~\cite{wassmer2024resilience}. However, since the networks considered here are typically on the scale of cities, including such distance-dependent factors does not substantially affect the results.


Panel (a) of \cref{fig:cost-gradient} shows the resulting traffic flows, while panel (b) presents the corresponding SCGC values for each edge. Centrally located edges tend to exhibit both higher traffic flows and larger SCGC values. This is because a greater number of shortest paths pass through these edges, making them critical for network-wide accessibility. Consequently, even small perturbations to their free flow travel times $\beta_e$ can strongly affect the total social cost.

Interestingly, five edges display negative SCGC values, indicating that increasing their free-flow travel times $\beta_e$ would \emph{decrease} the total social cost and thus identifying them as Braessian edges.

Panel (c) presents a scatter plot comparing traffic flows with SCGC values. While a positive correlation is evident, the relationship is not strictly linear or one-to-one. This demonstrates that SCGC captures additional structural information beyond simple traffic volumes, offering a more nuanced perspective on network vulnerability and intervention points.






%In \cref{fig:osm-flow}, we illustrate the convergence of user equilibrium flows under varying numbers of source nodes. Panel (a) compares six real-world road networks from German city districts—Nippes, Cologne (blue), Pankow, Berlin (orange), Potsdam (green), Heidelberg (red), Koblenz (purple), and Göttingen (brown)—all of similar size. To quantify convergence, we compute the difference in user equilibrium flows between a given number of source nodes, $|W|$, and the full solution with $|V|$ source nodes, then determine the standard deviation over all edges. Notably, this deviation increases rapidly across all graphs, indicating that using all nodes as source nodes may be unnecessary. To balance computational efficiency with accuracy, we find that selecting 10\% of nodes as source nodes provides a reasonable trade-off.\\

%Furthermore, to illustrate how source nodes are selected, panels (b–d) depict the user equilibrium flow on the road network of Nippes, Cologne, color-coded according to flow intensity. Panels (b) and (c) show results for 3 and 15 source nodes, respectively, with selected source nodes marked by red crosses. In panel (d), where all nodes act as source nodes, we omit the markers for clarity.



\subsection{Event-driven traffic in Cologne}\label{sec:football-cologne}
\begin{figure*}[ht]
    \centering
    \includegraphics[width=\textwidth]{assets/cologne-soccer.pdf}
    \caption{Visualization of vehicle flows and derivative social costs (SCGC) during a football event in Cologne, Germany. 
    Color coding in \textbf{(a)} shows vehicle flows $f_e$ across the road network. 
    Color coding in \textbf{(b)} depicts the derivative of the social cost with respect to $\beta_e$, highlighting Braess edges in blue.}
    \label{fig:football-cologne}
\end{figure*}
To illustrate the practical relevance of our framework, we analyze the traffic impact of a major football event in Cologne, Germany. 
We construct the city’s street network from OpenStreetMap (OSM) following the procedure in \cref{sec:generate_graphs}, resulting in a directed graph $G = (V, E)$ with $|V| = 835$ nodes and $|E| = 2{,}438$ edges. 
Each road segment is assigned a linear cost function as described in \cref{sec:osm-cost-func}, and travel demand is generated from population data according to \cref{sec:od-matrix}. 
This yields a realistic yet computationally tractable representation of the city’s road infrastructure and demand during large-scale events.


The primary source of traffic demand is the set of parking lots surrounding the football stadium. 
Their total capacity is approximately $15{,}000$ parking spaces, which corresponds to $n_\mathrm{vehicles} = 15{,}000 \times 2 = 30{,}000$ trips, assuming two trips per parking space over the course of the event. 
The parking locations are geocoded, and the nearest nodes in the network are defined as destination nodes $W$. 
We then solve the traffic assignment problem as described in~\cref{sec:tap-solution}, yielding user equilibrium flows $f_e$ on each edge. 
As shown in Figure~\ref{fig:football-cologne}\textbf{a}, the resulting flow pattern exhibits strong spatial concentration along the city’s main arterial roads leading to the stadium, which form the backbone of the event-related traffic system.\\

To assess the network’s vulnerability to inefficiencies, we compute the derivative of the social cost with respect to $\beta_e$, $\partial SC/\partial \beta_e$ (SCGC), for each edge as outlined in \cref{sec:scgc}. 
Edges with negative SCGC values indicate Braess edges—links where a decrease in capacity, for example by lowering speed limits (i.e., increasing free-flow travel time), could paradoxically reduce overall travel costs. 
Figure~\ref{fig:football-cologne}\textbf{b} highlights these edges in blue, revealing structurally important corridors and potentially counterproductive links whose management can substantially influence system performance.\\

Across the entire network of 2,438 edges (with a total length of 2,105\,km), we identify 33 Braess edges, i.e., about 1.3\% of all links, with a combined length of roughly 24.8\,km. 
The minimum SCGC value is $\min(\partial SC/\partial \beta_e) \approx -70$, indicating a strong potential for reducing total travel times through targeted interventions. 
In contrast, the largest positive SCGC value reaches $\max(\partial SC/\partial \beta_e) \approx 2{,}846$, concentrated along a few high-capacity arterial corridors. 
The median SCGC is relatively low (2.66), reflecting that only a limited number of edges exert a dominant influence on network efficiency.\\

Major arterial roads exhibit the highest positive SCGC values, reflecting their structural role in maintaining efficient access to the stadium. 
These roads are critical for event-related traffic and should remain as free of disturbances as possible. 
Measures such as restricting through traffic, imposing parking bans, or providing dedicated lanes could help safeguard their capacity and ensure smooth flows. 

In contrast, the identified Braess edges are primarily located on secondary or connector roads. 
Although they carry less flow, they have a disproportionately negative effect on network-wide performance. 
Reducing their attractiveness, e.g., through temporary speed limit reductions, dynamic access restrictions, or partial closures, can encourage more efficient routing and alleviate congestion elsewhere. 
Because these links are not central access routes to the stadium, such interventions can be implemented with minimal disruption to event traffic.\\

A promising traffic management strategy would be to prioritize access to high-SCGC (red) corridors for event traffic while diverting background traffic to lower-SCGC (blue or neutral) roads. 
This separation of flow classes can help prevent capacity bottlenecks and stabilize travel times. 
More generally, sensitivity-based indicators such as SCGC provide a systematic basis for identifying and managing structurally important or counterproductive links in urban road networks, enabling targeted, adaptive, and event-specific traffic control strategies.

Overall, this case study demonstrates how real-world data, traffic assignment modeling, and sensitivity analysis can be combined to produce actionable insights for urban mobility management. 
Rather than expanding capacity uniformly, targeted interventions on a small number of critical and Braess edges offer a cost-effective means of improving network performance during major events.



\subsection{Daily urban traffic analysis in Potsdam}\label{sec:potsdam}
\begin{figure*}[ht]
    \centering
    \includegraphics[width=\textwidth]{assets/Potsdam, Germany-braess.pdf}
    \caption{Visualization of vehicle flows and derivative social costs (SCGC) in Potsdam, Germany. 
    Color coding in \textbf{(a)} shows the edge utilization $1 - \beta_e / (\alpha_e f_e + \beta_e)$ across the road network, 
    while color coding in \textbf{(b)} depicts the derivative of the social cost with respect to $\beta_e$, highlighting Braess edges in blue.}
    \label{fig:potsdam}
\end{figure*}

To demonstrate the applicability of our framework to everyday mobility patterns, we analyze the daily traffic structure of Potsdam, Germany. 
The city’s street network is constructed from OpenStreetMap using the methodology described in \cref{sec:generate_graphs}, 
resulting in a directed graph $G = (V, E)$ with $|V| = 112$ nodes and $|E| = 296$ edges. 
Each edge is assigned a linear cost function following \cref{sec:osm-cost-func}, and travel demand is generated from population data as outlined in \cref{sec:od-matrix}. 
We choose the OD matrix such that the resulting flows represent a moderate traffic scenario, which is representative of typical conditions in many medium-sized European cities. 
In contrast to the Cologne scenario, where demand is concentrated near a stadium, here it is distributed evenly across the network to reflect typical daily mobility.\\

We begin by analyzing edge utilization, defined as
\begin{align*}
    u_e = 1 - \frac{\beta_e}{\alpha_e f_e + \beta_e}.
\end{align*}
This measure reflects the share of available capacity that is effectively used on a given edge: values close to one indicate heavily used links operating near capacity, while values close to zero correspond to underutilized links with little traffic relative to their capacity.  
Figure~\ref{fig:potsdam}a visualizes the resulting utilization pattern across the network. 
The average utilization is $0.46$, with a median of $0.46$ and a maximum of $0.94$, indicating a clear differentiation between heavily used corridors and underutilized links. 
We identify 26 edges (8.8\% of the network) with utilization above $0.8$, corresponding to a total length of 16.4\,km. 
These high-utilization corridors likely coincide with the city’s primary arterials, concentrating most of the daily traffic flow and playing a crucial role in maintaining network accessibility and efficiency.\\

At the same time, 48 edges (16.2\% of the network), covering 79.4\,km of roadway, exhibit utilization values below $0.2$. 
This underutilization highlights a large share of the network where capacity remains unused. 
However, low utilization alone does not necessarily imply low importance for overall network performance, since some low-flow links may still be structurally relevant. 
To determine their actual impact on efficiency, we turn to the SCGC analysis.\\

We next examine the derivative of the social cost with respect to $\beta_e$, $\partial SC/\partial \beta_e$ (SCGC), as described in \cref{sec:scgc}. 
Figure~\ref{fig:potsdam}b shows the spatial distribution of SCGC values, which range from $\min(\partial SC/\partial \beta_e) \approx -16.4$ to $\max(\partial SC/\partial \beta_e) \approx 1{,}117.8$, with a mean of $200.4$ and a median of $143.3$. 
Most edges exhibit positive SCGC values, indicating that increasing travel times on these links would negatively affect overall network performance. 
These edges largely overlap with the high-utilization corridors identified earlier, underscoring their central role in supporting efficient mobility.\\

In contrast, we identify only 2 Braess edges, with a total length of 1.5\,km. 
These links have negative SCGC values, meaning that increasing their free-flow travel times could paradoxically improve total network performance. 

In contrast to the Cologne case (see~\cref{sec:football-cologne}), where demand and supply are highly asymmetric, a more balanced spatial distribution of origins and destinations substantially reduces the likelihood of Braess edges emerging. When flows are dispersed more evenly across the network, the structural conditions that give rise to paradoxical effects become less pronounced.

While many of these edges exhibit relatively low utilization, it is their low or even negative SCGC values, rather than low flow alone, that identify them as non-critical for overall system efficiency. In other words, their marginal contribution to total social cost is small or beneficial when reduced. Such links therefore provide strategic flexibility for targeted interventions, including speed limit reductions, lane reallocation to cycling or bus infrastructure, or temporary closures during peak periods, without compromising aggregate network performance.
\\

This case study illustrates how combining utilization and SCGC analysis provides a more complete picture of network structure and function. 
High-SCGC and high-utilization corridors form the operational backbone of the system and should be protected to maintain stable traffic flow. 
Conversely, edges with low or negative SCGC values represent promising candidates for mode shift or other interventions that can promote more sustainable mobility without sacrificing efficiency. 
Potsdam’s network structure thus reveals both clear operational priorities and opportunities for targeted, low-impact interventions.


%%%%%%%%%%%%%%%%%%%%%%%%%%%%%%%%%%%%%
%%%%%%%%%%%%%%%%%%%%%%%%%%%%%%%%%%%%%
%%%%%%%%%%%%%%%%%%%%%%%%%%%%%%%%%%%%%
%%%%%%%%%%%%%%%%%%%%%%%%%%%%%%%%%%%%%
%%%%%%%%%%%%%%%%%%%%%%%%%%%%%%%%%%%%%

\clearpage

\section{Discussion and Conclusion}\label{sec:discussion}

This study proposed \emph{social cost gradient centrality} (SCGC), a novel sensitivity-based graph centrality for identifying critical and paradoxical edges in transportation networks. By linking marginal changes in link-level free-flow travel times to total social cost, SCGC provides a direct measure of how local infrastructure characteristics influence global network performance. Unlike traditional topological centrality measures, SCGC is grounded in the equilibrium flow structure of the network and reflects the emergent properties of self-organized traffic states.

\subsection{Interpretation and implications}

The results demonstrate that not all links contribute equally to system performance. A relatively small subset of edges exerts a disproportionately large influence on the overall social cost, highlighting their structural and functional centrality. Positive SCGC values indicate critical links whose degradation would significantly increase travel times, whereas negative values reveal Braessian edges where capacity expansion or travel time reductions would paradoxically worsen overall network efficiency. This provides a powerful diagnostic lens for understanding how local perturbations propagate through a self-organizing system.

From a conceptual standpoint, these findings resonate with the view of transport networks as complex adaptive systems in line with Haken’s principles of synergetics, where global order emerges from the interplay of a few dominant structures and many interacting components~\cite{haken1983}. In this framework, the few edges with high SCGC values act as ‘order parameters’ controlling the macroscopic flow patterns, reflecting the essence of Haken’s synergetic approach. Macroscopic traffic patterns emerge from decentralized decisions, and small local changes can produce nonlinear system-wide effects. SCGC effectively captures this local–global coupling by quantifying how the system responds to infinitesimal link-level interventions. This interpretation aligns with the principles of synergetics, where global order is shaped by the interplay of a few dominant structures and many interacting components.

\subsection{Policy relevance}

For urban mobility planning, the implications are twofold. First, the identification of critical edges enables targeted investment in infrastructure maintenance or enhancement where it has the greatest systemic benefit. Second, the detection of Braessian edges provides an analytical foundation for demand management, road space reallocation, or capacity reduction strategies that can improve overall efficiency despite appearing counterintuitive from a local perspective. These insights support the shift away from car-centric expansion strategies toward more intelligent network management and demand-side regulation.

Moreover, SCGC offers a quantitative tool for prioritizing measures such as congestion pricing, access restrictions, or adaptive signal control. By focusing on the links with the strongest impact on total social cost, policy makers can implement targeted and cost-effective interventions, avoiding broad, inefficient measures that overlook the network’s self-organizing structure.

\subsection{Limitations and future work}

While the current formulation provides a clear analytical foundation, several important limitations remain. A key restriction arises from the fact that SCGC exploits the piecewise linearity of total social cost with respect to free-flow travel times, which is inherent to the underlying linear link cost functions and static equilibrium formulation. This allows for an elegant and computationally efficient derivation of marginal sensitivities, but it also means that SCGC is strictly valid only within a fixed equilibrium regime. When marginal changes in link attributes lead to discontinuous shifts in route choice patterns or the activation of alternative paths, the derivative structure itself changes. Consequently, SCGC captures local sensitivity around the current equilibrium but does not provide a global picture of how larger interventions might reshape flow distributions and system performance.

This limitation is particularly relevant in networks with multiple near-optimal equilibria or where congestion dynamics are dominated by threshold effects. In such settings, even small infrastructure or demand changes can trigger qualitative shifts in traffic patterns that fall outside the linear response captured by SCGC. Future research should address this by extending the method to nonlinear cost functions or by embedding SCGC within dynamic traffic assignment frameworks that explicitly model time-dependent and path-switching behavior.

A second promising direction lies in broadening the scope of SCGC beyond single-layer, single-mode networks. Integrating multimodal and multilayer structures would enable the assessment of resilience and efficiency across more realistic urban mobility systems, where capacity interventions or disruptions can have cascading effects across modes.

Finally, coupling SCGC with control or optimization approaches represents a powerful avenue for policy applications. By identifying the links with the highest marginal impact, SCGC could inform targeted road pricing, capacity management, or resilience strategies. Extending the concept to capture higher-order interactions between edges, such as pairwise or groupwise sensitivities, would further enhance its ability to reveal the structural mechanisms through which local interventions propagate through complex transport networks.



\subsection{Conclusion}

This work advances the understanding of how local infrastructure characteristics shape global network behavior in self-organizing transport systems. SCGC provides a simple yet powerful tool to identify edges with disproportionate systemic importance, encompassing both critical and paradoxical effects. By grounding edge importance in network flows and social costs, this measure offers actionable insights for infrastructure planning, demand management, and policy design. SCGC operationalizes Haken's insight that global order in complex systems arises from local interactions and a few dominant components. More broadly, it demonstrates how sensitivity analysis can bridge microscopic infrastructure attributes and macroscopic system performance, contributing to the broader study of self-organization and control in complex networks.


\newpage
\appendix
\section{Link--path to node--link formulation}\label{app:link-path-node}
The traffic assignment problem (TAP) captures this behavior. Following~\citep{beckmann1956studies, dafermos1969traffic}, it can be formalized as an optimization problem
\begin{align}\label{eq:app-tap-link-path}
    \min_{{f_e}} \sum_{e \in E} \mathcal{F}(f_e) &, \nonumber \\
   \text{subject to } \sum_{r\in R^w} f_r^w &= d^w, \quad \forall w\in W, \nonumber \\
    f_r^w &\geq 0, \quad \forall r \in R^w, w \in W, \nonumber \\
    f_e &= \sum_{w\in W}\sum_{r\in R^w}\Delta_{e,r}^w f_r^w, 
\end{align}
where $f_e$ describes the flow on an edge $e$, while $f_r^w$ describes the flow on a path $r \in R$ between an origin-destination pair (OD) $w \in W$, such that the demand $d^w$ is covered.  Moreover, we define the link-path incidence matrix $\Delta^w_{e,r}$, which it is equal to one if edge $e$ lies on path $r$ between the OD $w$, and zero otherwise,
\begin{align}
\Delta^w_{e,r} = 
    \begin{cases}
        1, \text{ if edge $e$ is on path $r \in R^w$ of OD tuple $w$,}\\
        0, \text{ otherwise.}
    \end{cases}
\end{align}


The objective function $\mathcal{F}(f_e)$ for the user equilibrium of the TAP (see Eq.~\ref{eq:tap-node-link}) is given by the cumulative travel cost experienced by all drivers on each edge, which is represented by the integration of the cost function
\begin{align}
    \mathcal{F}_{\text{ue}}(f_e) &= \int_0^{f_{e}} t_{e}^{(k)}(u)du. \\
    &= \frac{t_e^0 f_e \left( (f_e)^k + k + 1\right)}{k+1}  \nonumber
\end{align}
The objective function for the system optimum of the TAP is given by the total travel time experienced by all drivers on a road combined (also known as the social cost), given by
 \begin{align}
     \mathcal{F}_{\text{so}}(f_e) = f_e t_e.
 \end{align}
Both objective functions are convex on the domain $[0, f_e^{\max}]$. Hence, the optimization problem~\eqref{eq:app-tap-link-path} can be solved in principal in polynomial time~\citep{kozlov1979polynomial}.
This formulation of the TAP is referred to as the \textit{link-path} formulation, as it requires the calculation of all paths from all origin-destination pairs. However, the number of all possible paths between the origin-destination pair $w$, generally increases exponentially with the size of a graph [Bollobas? Newman?], making this formulation of the TAP impractical to use.\\


To overcome the issue of having to compute all possible paths for a origin-destination pair, an alternative approach is to reformulate the TAP in a \textit{node-link} formulation. This can be achieved by defining the node-edge incidence matrix of a graph as 
\begin{align}\label{eq:app-edge-incidence}
    E_{ne} = 
    \begin{cases} 
    1 & \text{if edge } e \text{ originates at node } n, \\
    -1 & \text{if edge } e \text{ terminates at node } n, \\
    0 & \text{otherwise}.
    \end{cases}
\end{align}
Then, we can write down the TAP in the following way:
\begin{align}\label{eq:app-tap-node-link}
    \min_{{f_e}} \sum_{e \in E} \mathcal{F}(f_e) &, \nonumber \\
   \text{subject to } \sum_{e \in E} E_{ne}f^w_e &= p_n^w, \quad \forall n \in V, w \in W \nonumber \\
   f^w_e &\geq 0, \quad \forall e \in E, w \in W \nonumber,\\
   f_e &= \sum_{w\in W} f_e^w \quad \forall e \in E.
\end{align}
with 
\begin{align}
    \mathcal{F}(f_e) = 1/2 \alpha_ef_e^2  + \beta_e f_e.
\end{align}
where
\begin{equation}
    \begin{split}
    &p^w_n = \\
    &\begin{cases}
        d^w, & \text{ if node $n$ is the origin in a OD tuple $w$},\\
        -d^w, & \text{ if node $n$ is the destination in a OD tuple $w$},\\
        0, &\text{ otherwise}.
    \end{cases}   
\end{split}
\end{equation}
We will refer to the vector $\mathbf{p}^w$ with the elements $p^w_i, i \in V$ as source-sink vector.

\begin{lemma}[Equivalence of link-path and node-link constraints]
The constraints of the link-path formulation of the Traffic Assignment Problem (TAP) in Equation~\ref{eq:app-tap-link-path} can be equivalently transformed into the constraints of the node-link formulation of TAP in Equation~\ref{eq:app-tap-node-link}.
\end{lemma}
\begin{proof}
    We define the OD-pair-path incidence matrix $\Lambda_{rw}$ to be one if path $r \in R^w$ connects OD tuple $w$,
    \begin{align}
    \Lambda_{r}^w=
        \begin{cases}
            1, \text{ if path $r\in R^w$ connects OD tuple $w\in W$},\\
            0, \text{ otherwise.}
        \end{cases}
    \end{align}
Then, per definition the product of the node-edge incidence matrix and the edge-path incidence matrix is given by
\begin{align}
    &\sum_{e\in E} E_{ne}  \Delta^w_{e,r} =\\
    &=\begin{cases}
        1, & \text{ if $n$ is origin node of path $r \in R^w$},\\
        -1, & \text{ if $n$ is the terminal node of path $r\in R^w$},\\
        0, & \text{ otherwise.}
    \end{cases}\\
    &= \sum_{i \in V} \Lambda_{r}^{(n,i)} - \sum_{j \in V}\Lambda_{r}^{(j,n)}
\end{align}
Now, we use the definition of the flow over edge $e$ that emerges from the OD-pair $w$ which is given by
\begin{align}
    f_e^w &= \sum_{r\in R^w} \Delta^w_{er}f_r^{w}\\
    \text{such that } f_e &= \sum_{w\in W} f_e^w.
\end{align}
Thus,
\begin{align*}
    \sum_{e\in E} E_{ne}f_e^w &= \sum_{e\in E} E_{ne}  \sum_{r\in R^w} \Delta_{er}f_r^{w}\\
    &= \sum_{r\in R^w}  \left( \sum_{e\in E} E_{ne} \Delta_{er} \right) f_r^{w}\\
    &= \sum_{r\in R^w}  \left( \sum_{i\in V} \Lambda^{(n,i)}_r  f_r^{w}  - \sum_{j\in V} \Lambda^{(j,n)}_r f_r^{w} \right)\\
    &= \sum_{r\in R^w} \left(  \sum_{i\in V}  \delta_{w, (n,i)}f_r^{w}  -  \sum_{j\in V}  \delta_{w, (j,n)} f_r^{w} \right)\\
    &= \left( \sum_{r\in R^{(k,l)}} \sum_{i\in V}  \delta_{(k,l), (n,i)}f_r^{(k,l)} \right)\\
    & -\left( \sum_{r\in R^{(k,l)}}\sum_{j\in V}  \delta_{(k,l), (j,n)} f_r^{(k,l)} \right).
\end{align*}
In the last steps we used the definition of the OD-pair-path incidence matrix that is equal to one if the path $r$ originates at $o$ and terminates at $d$ for OD tuple $w$, which can be written as Kronecker delta $\delta_{w, (o,d)}$. In the next line we resolve the Kronecker deltas and use the definition of the demand constraint of Equation~\ref{eq:app-tap-link-path}.
\begin{equation}
    \begin{split}
    &\sum_{e\in E} E_{ne}f_e^w=\\ &=\sum_{r\in R^{(n,l)}}  f_r^{(n,l)}  - \sum_{r\in R^{(k,n)}}f_r^{(k,n)}\\
    &= d^{(n,l)} - d^{(k,n)}\\
    &= \begin{cases}
        -d^w, & \text{ if node $n$ is the origin in a OD tuple $w$},\\
        d^w, & \text{ if node $n$ is the destination in a OD tuple $w$},\\
        0, &\text{ otherwise},
    \end{cases}\\
    &= p_n^w.
    \end{split}
    \label{eq:conserve_flows}
\end{equation}
In the last step we used the source-sink constraint from Equation~\ref{eq:app-tap-node-link}.
Furthermore, the inequality constraints of Equation~\ref{eq:tap-node-link} can be transformed trivially into the inequality constraints of Equation~\ref{eq:tap-node-link}.
\end{proof}
\begin{lemma}[Flow decomposition theorem~\citep{ahuja1995network, krylatov2020optimization}]
    Every route flow $f_r$ has a unique representation as non-negative link flows $f_e$. [Conversely, every non-negative link flow may be represented as a route (though not necessary uniquely)].
\end{lemma}


\section{Constructing real-world networks}\label{sec:generate_graphs}
We construct directed graphs using road network data from the OpenStreetMap (OSM) project, leveraging the Python package OSMnx~\cite{boeing2025modeling}. OSM provides detailed geographical and infrastructural information about roads in a specified region, including their location, length, speed limit, number of lanes, and other metadata.\\

Real-world graph networks can become very large. For example, consider Berlin, the largest city in Germany: its road network comprises $N = 5{,}711$ nodes and $M = 11{,}882$ edges. If every node were allowed to act as an origin in the traffic assignment problem (TAP), i.e., $|W| = 5{,}711$, the resulting system would involve $5{,}711 \times 11{,}882$ flow variables and $5{,}711 \times 5{,}711$ constraints. Solving such a system directly is computationally infeasible.\\

To address this challenge, we apply a series of spatial and topological simplification steps that preserve the essential structure of the network while making the problem tractable.\\

First, we exclude minor roads, in particular residential streets, which contribute little to large-scale traffic flow but significantly increase network size and complexity. This step effectively filters the graph to include only major road classes such as motorways, trunk roads, and primary or secondary streets.\\

Second, we use the \texttt{osmnx.simplification.consolidate\_intersections} function to merge nearby nodes representing the same physical intersection. Divided roads, roundabouts, or complex intersections are often represented in OSM by multiple closely spaced nodes, which unnecessarily inflate the number of variables and constraints in the TAP formulation. Consolidating these nodes reduces network complexity while preserving its essential connectivity and flow structure. The tolerance parameter for node consolidation is set to $100\,\mathrm{m}$, reflecting typical urban intersection spacing.\\

Finally, we restrict the set of destination nodes $W$ to a strategically selected subset that captures the essential travel demand structure. For example, we may consider centroids of administrative districts, major transportation hubs, or points of interest such as football stadiums. By focusing on a representative set of destinations rather than the full node set, we drastically reduce the dimensionality of the problem while retaining meaningful large-scale flow patterns.\\

These preprocessing steps yield a simplified yet topologically representative network, enabling efficient computation of user equilibrium flows on real-world road networks without sacrificing key structural characteristics.


\section{Accessing the cost function}\label{sec:osm-cost-func}
To compute the cost function (see \cref{eq:linear-travel-time}), we infer the parameters $\alpha_e$ and $\beta_e$ for each road segment $e \in E$. These parameters are derived directly from the physical and infrastructural characteristics of the road network obtained from OpenStreetMap.\\

The free-flow travel time $\beta_e$ is determined from the length $l_e$ of the road segment and its speed limit $v_e$, ensuring that baseline travel times reflect realistic traffic conditions under uncongested flow:
\begin{align*}
    \beta_e = \frac{l_e}{v_e}.
\end{align*}

To represent congestion effects, we compute the congestion factor $\alpha_e$ using the road length, speed limit, and number of lanes $m_e$. Additionally, we define the edge capacity $x^{\max}_e$ and the maximum travel time $t^{\max}_e$ based on a reference walking speed, which acts as an upper bound in extreme congestion scenarios.\\

The congestion factor $\alpha_e$ is given by
\begin{align*}
    \alpha_e = \frac{t^{\max}_e - \beta_e}{5\,x^{\max}_e},
\end{align*}
where $t^{\max}_e = \frac{l_e}{v_{\mathrm{walk}}}$ represents the travel time at walking speed and $x^{\max}_e$ is the effective capacity of the road segment. The scaling factor $5$ controls the slope of the cost function and can be tuned to adjust how quickly travel times increase with flow.



\section{Constructing the OD matrix}\label{sec:od-matrix}
To generate the OD matrix $p_n^w$, we use the population distribution of the region under study. Specifically, we rely on the Global Human Settlement Layer (GHSL)~\cite{GHSL2023} to assign a population count to each node in the network. This is achieved by computing Voronoi polygons around the nodes and intersecting them with the GHSL raster, allowing us to allocate population values proportionally to their spatial influence area (see~\citep{wassmer2024resilience, wassmer2025unveiling} for a detailed explanation of this process).\\

Based on these node-level population values, we assume that the likelihood of trips originating from or traveling to a node is solely determined by its population. As a result, the probability of traveling from one node to another is independent of the distance between them. While more sophisticated approaches could incorporate distance-based effects—such as a gravity model—this simplification is justified by the typical spatial scale of our networks (urban areas), where trip length distributions are relatively homogeneous and distance effects are less pronounced.\\

The OD matrix is generated by distributing a fixed fraction $\gamma$ of each node’s population across a selected set of destination nodes in proportion to their population. This results in
\begin{align*}
    p_n^w = \gamma \, \frac{P_w}{\sum_{w' \in W} P_{w'}} P_n,
\end{align*}
where $P_n$ denotes the population at origin node $n$ and $P_w$ the population at destination node $w$. 








\bibliography{references}
\end{document}
